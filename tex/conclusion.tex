\chapter{Conclusion}
\section{Discussion}
In this work, we have explored various contemporary attempts at replicating \acfp{ASC} induced by classical psychedelics. We have also explored potential applications of the replications of psychedelic-induced \acp{ASC}.

We created an application for immersive \acf{VR}, which is capable of producing a complex replication of a psychedelic-induced \ac{ASC}. This complex replication was a combination of partial replications of the following distincts aspects of \acp{ASC}:

\begin{enumerate}
    \setlength{\itemsep}{0pt}
    \setlength{\parskip}{0pt}
    \item Depth Perception Distortion
    \item Visual Drifting
    \item Visual Acuity Enhancement
    \item Tracers (Visual Tracing)
\end{enumerate}

We measured the impact of the implemented complex replication with $N=10$ subjects on the scores of the \acf{5D-ASC} and the \acf{11-ASC}, and confirmed, that they are influenced by the implemented complex replication. This finding suggests, that \ac{VR} is an effective medium for the replication of certain aspects of \acp{ASC}.

\section{Limitations}
The study we have conducted was constrained by the following limitations.

\textbf{Lack of the Visualization of the User's Body}.\\
Displaying the user's body in \ac{VR} realistically is an unsolved issue for consumer-grade \ac{VR} systems. Ideally, the user would be able to see their real body, possibly via a filtered view of a pass-through camera built-into the \acf{HMD}, or maybe via a 3D-reconstructed model from external cameras. A 3D-reconstructed model would be ideal, because the spatial effects could be applied to the scene as well as the body, which would provide an better sense of micropsia and macropsia.

\textbf{Low Sample Size.}\\
The results of our study were mainly limited by the low number of participants ($N=10$). We might have been able to reach more conclusive results, had the sample size been larger.

\textbf{No Sound Isolation.}\\
Our experimental conditions did not allow for sound isolation. Often, noises could be heard from adjacent rooms, or from the outside through the window that we consistenly kept open during testing. In retrospect, a closed room with proper sound isolation and a quiet ventilation system would have provided better experimental conditions. The silence could be substituted with natural ambient noises appropriate for the virtual scene.

\section{Future Work}
Other aspects (time perception, synaesthesia), stronger hypothesis, eye tracking, hand tracking, body tracking, indoor scene, improvements for tracer effect, geometry/mesh shaders
