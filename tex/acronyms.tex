\chapter*{List of Acronyms}
\vspace{-1.6em}

\begin{acronym} % keep sorted using :sort
    \acro{11-ASC}{11-Factor Altered States of Consciousness Questionnaire\acroextra{: A version of the \textit{Altered States of Consciousness Rating Scale} psychometric questionnaire, which is based on the hypothesis that \acp{ASC} have a common core independent of the induction method which distinguishes them from the waking conscious state \autocites{figueiredobuilding}{studerus2010psychometric}.}}
    \acro{5-HT}{5-hydroxytryptamine, also known as serotonin}
    \acro{AI}{artificial intelligence}
    \acro{ASC}{altered state of consciousness\acroextra{: See section \ref{sec:asc_definition} for a complete definition and related terms.}} \acrodefplural{ASC}{altered states of consciousness}
    \acro{DCNN}{deep convolutional neural networks}
    \acro{DMT}{\textit{N,N}-dimethyltryptamine\acroextra{: A classical hallucinogenic drug first synthesized in 1931 \autocite{manske1931synthesis}, a psychoactive compound of Ayahuasca, the ceremonial spiritual medicine used by Amazonian natives for shamanic purposes and to bond socially in a casual setting \autocite{hay2020ayahuasca}.}}
    \acro{DSP}{digital signal processing}
    \acro{EEG}{electroencephalograph}
    \acro{FOV}{field of view}
    \acro{GPU}{graphics processing unit\acroextra{: A specialized extension module providing acceleration for computer graphics computations and other parallelizable tasks.}}
    \acro{HMD}{head--mounted display}
    \acro{LSD}{lysergic acid diethylamide\acroextra{: A classical hallucinogenic drug first synthesized in 1938 from ergotamine, an alkaloid of the ergot rye fungus \autocite{hofmann1969lsd}.}}
    \acro{MEQ30}{30-item revised mystical experience questionnaire}
    \acro{MTE}{`mystical-type' experience\acroextra{: Subjective experiences whose characteristics include a sense of connectedness, transcendence, and ineffability.}}
    \acro{PCI}{Phenomenology of Consciousness Inventory\acroextra{: A psychometric questionnaire based on the hypothesis that different states of consciousness can be characterized in terms of phenomenological dimensions which can be quantified in terms of their intensity. The resulting pattern is assumed to be typical of a particular induction method and can be observed consistently \autocite{figueiredobuilding}.}}
    \acro{VR}{virtual reality}
\end{acronym}
