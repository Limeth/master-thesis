\chapter{Introduction}
\vspace{-1.6em}
\hspace{57pt}{\large\textbf{Contents}}%

\minitoc
\thispagestyle{empty}
\newpage

\section{Problem Statement}
This thesis is focused on the development of a \ac{VR} application that simulates select aspects of \acep{ASC}{further defined in \ref{sec:asc_definition}} typically induced by classical psychedelics such as LSD and psilocybin/psilocin.
We focus primarily on the recreation of the \acp{ASC}' effects on sensory perception using an analytical approach.

\section{Motivation}
Due to their high degree of immersion, \ac{VR} systems, with \acp{HMD} in particular, offer a unique opportunity for recreating certain aspects of \acp{ASC}.

\subsection{Art and Media}
\Acp{ASC} of various forms have had a significant influence on art for millenia. Earliest signs of inductions of \acp{ASC} via neurotropic substances have been found possibly as early as 60,000 BC \autocite{guerra2015psychoactive}. \Acp{ASC} continue to be depicted in or influence contemporary popular media.

An analysis and a recreation of certain aspects of \acp{ASC} may serve as a reference point for recreating those aspects of \acp{ASC} in popular media.

\subsection{Education}
While experiencing a simulation of an \ac{ASC} is unlikely to be fully representative of the \ac{ASC} the simulation is modelled after, we propose that the simulation may be significantly less inducive of difficult experiences coloquially known as ''bad trips''.

This may be a viable alternative form of experiencing certain aspects of \acp{ASC}, while the possession or consumption of mind--altering substances is illegal in most countries. The resulting \ac{VR} application may serve as an educational tool about \acp{ASC} which would not require as controlled of an environment as is required in psychedelic--assisted psychotherapy.

\subsection{Research and Psychotherapy}
\textcite{aday2020psychedelics} make an interesting observation, that psychedelics and VR are utilized in tandem to enhance the experience of recreational users. Moreover, the authors claim that VR could also be used to optimize and tailor the therapeutic setting during psychedelic sessions.

Most importantly, however, the authors state, that:

\begin{quote}
    [...] VR may be a useful tool for preparing hallucinogen-naïve participants in clinical trials for the sensory distortions experienced in psychedelic states.
\end{quote}

Nonetheless, as mentioned previously, care should be taken to ensure that users experiencing the simulation are informed about the simulation not being fully representative of the \ac{ASC} it is modelled after. While a \ac{VR} simulation may be suitable for simulating sensory effects of \acp{ASC}, other effects, such as the effects on cognition, may be much more difficult, if not impossible, to directly replicate via \ac{VR} technologies alone. If uninformed, users may gain a false impression about the \ac{ASC}.

\subsection{Understanding of the Human Mind}
Finally, without any immediate application, the study of the effects of \acp{ASC} may help contribute to our understanding of the human mind. Particularly, for instance, analyzing the invariant effects of classical psychedelics on sensory perception may improve our understanding of the visual cortex and the way it functions.
Further research involving perceptual phenomena and pharmacodynamics of psychedelics and their mechanisms of action may contribute to our understanding of the significance of certain receptors in processing visual or other sensory information.

\section{Related Work}

\subsection{Recreations of Visual Phenomena}
In this section, we explore the way \acp{ASC} have been depicted in contemporary art and media and recent attempts at recreating aspects of \acp{ASC} in the scientific domain.

\subsubsection{Quake Delirium}
In the original paper about Quake Delirium \autocite{weinel2011quake}, the author divides video games portraying \acp{ASC} into two categories:

\begin{enumerate}
    \item Games which feature literally portrayed dreams, intoxication or hallucinogenic experiences.
    \item Games which feature graphical or thematic content which audiences may consider to reflect states of dream, intoxication or hallucination, but without any direct or literal reference to these states.
\end{enumerate}

This categorization is not only useful for examining video games, but also the rest of art and media.

The first category describes media that attempts to recreate \acp{ASC} with an explicit reference to a specific induction method, cause or origin. For example, the games in this category, such as {Grand Theft Auto: Vice City}\footnote{Grand Theft Auto: Vice City, Rockstar Games, 2003. PC (Windows) CD-ROM.} or {Duke Nukem 3D}\footnote{Duke Nukem 3D, 3D Realms, 1996. PC (Windows)
CD-ROM.}, may temporarily portray a character under the influence of a psychoactive drug. However, psychoactive substances are not the only form of \ac{ASC} induction portrayed in video games. One such exception is {LSD: Dream Emulator}\footnote{LSD: Dream Emulator, Asmik Ace Entertainment, 1998. Playstation.}, a game with narrative based on a dream diary and an overall dream-like surrealist aesthetic.

The second category contains media that does not communicate explicitly any \ac{ASC} method of induction, cause or origin. Despite this, the media that falls into this category may be viewed equally as \textit{psychedelic} or more than that of the first category. This could be considered to be the case of {Yoshi's Island}\footnote{Super Mario World 2: Yoshi’s Island, Nintendo, 1995. Super NES.}. While the creators may not have intended the video game to reflect \acp{ASC}, because of it's brightly colored surrealistic visual themes, it may resemble \acp{ASC} of games from the first category.

The \textit{Quake Delirium} project itself is a modification of the game \textit{Quake} that makes use of an external \ac{DSP} audio patch for modifying the resulting audiovisual output the game produces. The visual effects consist of changes in:

\begin{enumerate}
    \item \ac{FOV};
    \item camera swaying;
    \item fog density and color;
    \item game speed;
    \item stereo vision (for 3D red cyan glasses);
    \item gamma;
    \item hue.
\end{enumerate}

These visual effects are made available to the \ac{DSP} patch, the control of which can be automated using multi-track audio sequencing software. This enables the effects to onset slowly and gradually become more severe over time.

The project demonstrates a method of combining multiple partial effects that results in a complex audio-visual effect that is more sophisticated than many of the existing games exhibiting phenomena of \acp{ASC} surveyed.

\subsubsection{Crystal Vibes}
\autocite{outram2017crystal}
tactile stimulation further discussed in \ref{sec:tactile_stimulation_interfaces}

\subsubsection{Isness}
\autocite{glowacki2020isness}

\subsubsection{Hallucination Machine}
\autocite{suzuki2018hallucination}

\subsection{Tactile Stimulation Interfaces}\label{sec:tactile_stimulation_interfaces}

\subsubsection{Synesthesia suit for Rez Infinite}
(\textcite{konishi2016synesthesia1}, \textcite{konishi2016synesthesia2} and \textcite{synesthesia2016suit})
further improved by \autocite{furukawa2019synesthesia}

\subsubsection{Synesthesia X1 - 2.44}
\autocite{synesthesia2021x1}

\subsubsection{Subpac}
(\textcite{subpac2013subpac}, \textcite{drempetic2017wearable})
used in \autocite{zimmermann2016longing}
and studied on deaf people \autocite{schmitz2020hearing}


\section{Contributions}
We develop a \ac{VR} application for \acp{HMD} that simulates select aspects of \acp{ASC}.
We perform a study in which we measure the influence of the created \ac{VR} application on the human mind. This measurement is done via the \ace{11-ASC}{\textcite{studerus2010psychometric}}, that is used in clinical studies of psychedelic drugs.
