\chapter{Evaluation}

\DeclareRobustCommand\makeresultstable[3]{
    \pgfplotstableread[col sep=tab]{#2}#1

    \pgfplotstablegetrowsof{#1}
    \pgfmathsetmacro{\N}{\pgfplotsretval-1}

    \def\barspacing{2pt}

    \begin{tikzpicture}
        \begin{axis}[
            xlabel=\% of theoretical maximum,
            xmajorgrids,
            enlarge y limits={abs=0.75cm},
            legend style={
                at={(0.5,0)},
                anchor=south,
                yshift=-2cm,
                legend columns=-1,
                /tikz/every even column/.append style={column sep=0.25cm},
            },
            xmin=-5,
            xmax=30,
            xbar=\barspacing,
            bar width=7pt,
            ytick=data,
            yticklabels from table={#1}{Title},
            y tick label style={
                yshift=0.1cm,
                xshift=-0.1cm,
                anchor=east,
                rotate=45,
                font=\footnotesize,
                % font=\scriptsize,
            },
            y dir=reverse,
            reverse legend,
            cycle list={
                {red,fill=red!30!white,mark=none},%
                {blue,fill=blue!30!white,mark=none},%
                {black,fill=black!30!white,mark=none},%
            },
            after end axis/.code={
                \foreach \i in {0,...,\N}{
                    \pgfplotstablegetelem{\i}{Significance}\of#1
                    \let\curr\pgfplotsretval
                    \pgfplotstablegetelem{\i}{Mean_Test}\of#1
                    \let\currmeantest\pgfplotsretval
                    \pgfplotstablegetelem{\i}{Mean_Control}\of#1
                    \let\currmeancontrol\pgfplotsretval
                    \pgfplotstablegetelem{\i}{Confidence_Interval_Test}\of#1
                    \let\currconfinttest\pgfplotsretval
                    \pgfplotstablegetelem{\i}{Confidence_Interval_Control}\of#1
                    \let\currconfintcontrol\pgfplotsretval
                    \pgfmathsetmacro{\yshift}{\fpeval{0.5*{\barspacing+0.5*\pgfkeysvalueof{/pgf/bar width}}}}
                    \if\instring{*}{\curr}
                        \StrLen{\curr}[\currast]
                        \pgfmathsetmacro{\xvaluemax}{26.25}
                        \pgfmathsetmacro{\xvaluemincontrol}{\fpeval{1+\currmeancontrol+\currconfintcontrol}}
                        \pgfmathsetmacro{\xvaluemintest}{\fpeval{1+\currmeantest+\currconfinttest}}
                        \draw
                            ([yshift=-\yshift]axis cs:\xvaluemintest,\i) --
                            ([yshift=-\yshift]axis cs:\xvaluemax,\i) --
                            node[right, yshift=-0.7mm]{\curr}
                            ([yshift=\yshift]axis cs:\xvaluemax,\i) --
                            ([yshift=\yshift]axis cs:\xvaluemincontrol,\i);
                    \fi
                }%
            },
            #3
        ]
            \draw (0,-100) -- (0,100);

            \addplot+ [
                error bars/.cd,
                x dir=both,
                x explicit,
            ] table [
                y expr=\coordindex,
                x=Mean_Test,
                x error=Confidence_Interval_Test,
            ]{#1};

            \addplot+ [
                error bars/.cd,
                x dir=both,
                x explicit,
            ] table [
                y expr=\coordindex,
                x=Mean_Control,
                x error=Confidence_Interval_Control,
            ]{#1};

            \legend{Test,Control}
        \end{axis}
    \end{tikzpicture}
}

\section{Methods}
We propose the following hypotheses to investigate whether the implemented complex replication has any effect on the scores of the \ac{5D-ASC} and \ac{11-ASC} scorings.

\begin{itemize}
    \item Null hypothesis $H_0$: The implemented complex replication \textbf{does not have any} effect on the scores of the \ac{5D-ASC} and \ac{11-ASC} scorings.
    \item Alternative hypothesis $H_1$: The implemented complex replication \textbf{does have an} effect on the scores of the \ac{5D-ASC} scoring, the \ac{11-ASC} scoring, or both.
\end{itemize}

A controlled within-subject study including $N=10$ participants, was conducted to test these hypotheses, with a statistical significance level of $\alpha=0.05$.

The following demographic statistics are known about the population:
\begin{itemize}
    \setlength{\itemsep}{0pt}
    \setlength{\parskip}{0pt}
    \item Age: mean of 30 years, standard deviation of 10.1 years.
    \item Gender: 5 male, 4 female, 1 non-binary.
\end{itemize}

The requirements for participation in the study, inspired by \textcite{bartossek2021altered}, were as follows:
\begin{itemize}
    \setlength{\itemsep}{0pt}
    \setlength{\parskip}{0pt}
    \item No alcohol use in the past 12 hours prior to the session.
    \item No \acs{THC} use in the past week prior to the session and no more than twice a week in the last year.
    \item No use of psychedelic substances in the last two weeks prior to the session.
    \item Not pregnant.
    \item 18 years or older.
\end{itemize}

The participants were required to reserve two sessions, each on a different day, for the active and control scenarios, the order of which was chosen at random. 5 out of 10 participants underwent the control scenario on their first session.

The first session consisted of:
\begin{enumerate}
    \setlength{\itemsep}{0pt}
    \setlength{\parskip}{0pt}
    \item An explanation of the study and the procedure of the session.
    \item Signing of the informed consent form (see appendix \ref{appendix:informed-consent}).
    \item Testing of the developed VR application for 10 minutes, with either the control or the test scenario.
    \item Filling out of the psychometric questionnaire.
\end{enumerate}

The second session consisted of:
\begin{enumerate}
    \setlength{\itemsep}{0pt}
    \setlength{\parskip}{0pt}
    \item Testing of the developed VR application for 10 minutes, with the remaining scenario.
    \item Filling out of the psychometric questionnaire.
\end{enumerate}

The study was conducted with the following hardware specifications.

\begin{center}
\begin{tabular}{r l}
    \acs{HMD} & HTC Vive Pro \\
    Tracking & 2 SteamVR Base Stations 2.0 \\
    \acs{CPU} & Intel\textregistered\ Core™ i9-10900X \\
    \acs{GPU} & NVIDIA\textregistered\ GeForce\textregistered\ RTX 2080 Ti \\
    \acs{RAM} Capacity & 128 GB \\
\end{tabular}
\end{center}

\newpage % without this, the pages are stretched

\section{Results}

With the assumption that the individual factors of each of the scorings are normally distributed, we used the repeated measures \textit{paired t-test} to test our hypotheses. Most recent studies utilizing the \ac{5D-ASC} and the \ac{11-ASC} scorings assume the resulting factors to be normally distributed, and so do we. For alternatives, see appendix \ref{appendix:analysis-remarks}.

The results of the \ac{5D-ASC} scorings, shown in figure \ref{fig:results-5d-asc}, indicate a significant difference for the ``Dread of Ego Dissolution`` ($p = 0.0204 < 0.05$) and ``Vigilance Reduction'' ($p = 9.16 \cdot 10^{-6} < 0.001$) dimensions.

The results of the \ac{11-ASC} scorings can be seen in figure \ref{fig:results-11-asc}, and indicate a significant difference for the ``Disembodiment'' ($p = 0.0357 < 0.05$) and ``Anxiety'' ($p = 0.0387 < 0.05$) factors.

Based on these results, we \textbf{reject} the null hypothesis $H_0$, that ``the implemented complex replication \textbf{does not have any} effect on the scores of the \ac{5D-ASC} and \ac{11-ASC} scorings.''

\begin{figure}
    \centering
    \ifgraphics
        \makeresultstable{\resultstablefivedasc}{data/results_5d_asc.csv}{width=10cm,height=7cm}
    \fi
    \caption{
        Resulting scores according to the \ac{5D-ASC} scoring for control and test scenarios.
        Error bars show the 95\% confidence interval of the true mean, assuming normal distribution.
        Statistical significance indicated with \raisebox{-0.7ex}{*} ($p < 0.05$) and \raisebox{-0.7ex}{***} ($p < 0.001$).
    }
    \label{fig:results-5d-asc}
\end{figure}

\begin{figure}
    \centering
    \ifgraphics
        \makeresultstable{\resultstablefivedasc}{data/results_11_asc.csv}{width=10cm,height=12cm}
    \fi
    \caption{
        Resulting scores according to the \ac{11-ASC} scoring for control and test scenarios.
        Error bars show the 95\% confidence interval of the true mean, assuming normal distribution.
        Statistical significance indicated with \raisebox{-0.7ex}{*} ($p < 0.05$).
    }
    \label{fig:results-11-asc}
\end{figure}
