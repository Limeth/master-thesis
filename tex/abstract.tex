\chapter*{Abstract}

{\Large\textbf{English}}\\
In this thesis, I explore contemporary attempts at the replication of various aspects of psychedelic-induced altered states of consciousness (ASCs) via software; primarily, the phenomena related to visual perception. Then, I describe my implementation of a replication, that simulates select aspects of a low-dose ASC induced by classical, serotonergic psychedelics, using immersive virtual reality (VR). I describe a study that I have conducted in order to measure the impact of the resulting implementation on the results of psychometric questionnaires (5D-ASC, 11-ASC) typically used in the evaluation of subjective effects in clinical studies of psychedelics.

\textbf{Keywords}: altered state of consciousness, virtual reality, simulation, psychedelics, hallucinogens, phenomenology, replication.

\vspace{0.5cm}
\noindent
{\Large\textbf{Česky}}\\
V této práci prozkoumávám současné pokusy o replikaci různých aspektů psychedeliky-vyvolaných pozměněných stavů vědomí (ASC) pomocí softwaru; především fenoménů souvisejících se zrakem. Poté popisuji svou implementaci replikace, která simuluje vybrané aspekty ASC vyvolané nízkou dávkou klasických, serotonergních psychedelik, pomocí imerzivní virtuální reality (VR). Popisuji studii, kterou jsem provedl pro změření vlivu výsledné implementace na výsledky psychometrických dotazníků (5D-ASC, 11-ASC), které se typicky používají pro ohodnocení subjektivních efektů v klinických studiích psychedelik.

\textbf{Klíčová slova}: pozměněný stav vědomí, virtuální realita, simulace, psychedelika, halucinogeny, fenomenologie, replikace.
